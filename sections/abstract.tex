\abstract{
Anomaly detection is used to discover and understand network security issues, especially pertaining to distinguishing between normal and attack traffic. Since a single entity has a limited vantage point from which to observe network traffic, the sharing of network traffic data has been proposed as a method to improve the performance of network anomaly detection. Existing work has largely focused on sharing characteristics of detected anomalies. In this paper, we take a broader view of network data sharing and aim to characterize when it can improve anomaly detection performance. In particular, we focus on the sharing of network traffic data and the models trained on them in the supervised, unsupervised and semi-supervised learning paradigms. For each mode of learning, we consider the impact of the volume of data, its heterogenity and labeling errors on the performance of anomaly detectors. Our experiments on 2 publicly available and 1 curated dataset, using 5 different types of models indicate that:
\TODO{add experimental conclusions here}}
